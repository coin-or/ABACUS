\section{New Features of ABACUS 2.0}
\label{section:new20}

This section summarizes all new features that have been introduced
since the release of \ABACUS\ 1.2. 

\subsection{LP-Solver Soplex}
\index{Soplex}

Besides Cplex \ABACUS\ provides now an interface to the LP-Solver
Soplex \cite{Wun97}.

If Soplex is used as LP-solver, it might be
required to switch to the
new include file structure (see
Section~\ref{section:includeFilePath}) in order to avoid name
conflicts. Both Soplex and \ABACUS\ provide include files with the
name \texttt{timer.h}.

\subsection{Naming Conventions}
\label{section:namingConventions}

The previous version did not use any prefix for all globally visible
names in order to avoid name collisions with other libraries since the
\CPLUSPLUS\ concept of namespaces should make this technique
redundant. Unfortunately, it turned out that the GNU \CPLUSPLUS\
compiler does still not support namespaces. The \GPLUSPLUS-FAQ
mentions that even in the next release 2.8 this concept might not be
supported.

In order to provide the possibility of
avoiding name collisions without namespaces,
we added to all globally visible names the prefix \texttt{ABA\_}.
There are two possibilities for reusing your old codes together with
the new name concept. 

The first method is to include the file {\tt oldnames.h} into every
file using \ABACUS\ names without the prefix {\tt ABA\_}. 
In the compilation the preprocessor flag 
{\tt ABACUS\_OLD\_NAMES}\index{ABACUS\_OLD\_NAMES@{\tt ABACUS\_OLD\_NAMES}}
must be set. With
preprocessor definitions the old names are converted to new names. 
You should be aware that this technique can have dangerous side
effects. Therefore, this procedure should {\bf not} be applied if you combine
\ABACUS\ with any other library in your application.

The second method is the better method and is not much more work than
the first one. In the {\tt tools} subdirectory of the \ABACUS\
distribution you can find the Perl script 
{\tt upd-names-2.0}\index{upd-names-2.0@{\tt upd-names-2.0}}.
If you apply this script to all source files of your \ABACUS\
application by calling
\begin{verbatim}
  upd-names-2.0 <files>
\end{verbatim}
a copy of each file given in {\tt <files>} is made in the subdirectory
{\tt new-files} and the old names are replaced by the new names. 

\subsection{Include File Path}
\label{section:includeFilePath}

Another problem are header files of different
libraries with the same name. It can happen that due to the inclusion
structure it is not possible to avoid these conflicts by the order 
of the include file search paths. Therefore, every \ABACUS\ include
file ({\tt *.h} and {\tt *.inc}) is included now from the subdirectory 
{\tt abacus}. You can continue using the old include file structure by
setting the preprocessor flag 
{\tt ABACUS\_OLD\_INCLUDE}\index{ABACUS\_OLD\_INCLUDE@{\tt ABACUS\_OLD\_INCLUDE}}. Here is an
example how an \ABACUS\ file includes other \ABACUS\ files:
\begin{verbatim}
#ifdef ABACUS_OLD_INCLUDE
#include "array.h"
#else
#include "abacus/array.h"
#endif
\end{verbatim}
We strongly recommend the use of the new include file structure.
In combination with the LP-solver Soplex the new include file
structure is sometimes required (it depends which \ABACUS\ and
which Soplex files you include). There may be name conflicts since both
systems have a file {\tt timer.h}.

Due to this concept also the directory structure of the \ABACUS\
distribution has changed. All include files are now in the
subdirectory {\tt include/abacus}. 

A conversion can be performed with the help of the Perl script 
{\tt tools/upd-includes-2.0}\index{upd-includes-2.0@{\tt upd-includes-2.0}}.
 Calling
\begin{verbatim}
  upd-includes-2.0 <files>
\end{verbatim}
makes a copy of all {\tt <files>} into the subdirectory 
{\tt new-includes} and adapts them to the new include structure, e.g.,
\begin{verbatim}
  #include "master.h"
\end{verbatim}
is replaced by
\begin{verbatim}
  #include "abacus/master.h"
\end{verbatim}
in the new files.

\subsection{Advanced Control of the Tailing Off Effect}
\index{tailing off}

\ABACUS\ automatically controls the tailing off effect according to the
parameters {\tt TailOffNLps} and {\tt TailOffPercent} of the
configuration file {\tt .abacus}. Solutions of LP-relaxations can be
skipped in this control by calling the function {\tt
  ignoreInTailingOff()}\index{ABA\_SUB@{\tt ABA\_SUB}!ignoreInTailingOff@{\tt ignoreInTailingOff}}
(see Section~\ref{section:ignoreInTailingOff}).


\subsection{Problem Specific Fathoming}
\index{fathoming!problem specific}

Problem specific fathoming criteria can be added by the redefinition 
of the virtual function 
{\tt ABA\_SUB::ex\-cep\-tion\-Fathom()}\index{ABA\_SUB@{\tt ABA\_SUB}!exceptionFathom@{\tt exceptionFathom}} (see
Section~\ref{section:exceptionFathom}).

\subsection{Problem Specific Branching}
\index{branching!problem specific}

A problem specific branching step can be enforced by the redefinition
of the virtual function 
{\tt ABA\_SUB::ex\-cep\-tion\-Branch()}\index{ABA\_SUB@{\tt ABA\_SUB}!exceptionBranch@{\tt exceptionBranch}} (see Section~\ref{section:exceptionBranch}).

\subsection{Generalized Strong Branching}
\index{strong branching}

Generalized strong branching is the possibility of evaluating
different branching rules and selecting the best ones. If
branching on variables is performed, e.g., the first linear programs of the
(potential) sons for various branching 
variables are solved, in order to find the
most promising variable. Together with the built-in branching strategies
this feature can be controlled with the new entry 
{\tt NBranch\-ing\-Vari\-able\-Can\-di\-dates}\index{NBranchingVariableCandidates@{\tt
  NBranchingVariableCandidates}} of the configuration file 
(Section~\ref{section:parameters}).
Moreover, also other branching strategies can be evaluated as
explained in Section~\ref{section:StrongBranching}.
        

\subsection{Pool without Constraint Duplication}
\index{pool!without duplication}

One problem in using \ABACUS\ can be the large number of generated
constraints and variables that use a lot of memory. In order to reduce
the memory usage we provide a new pool class 
{\tt ABA\_NONDUPLPOOL}\index{ABA\_NONDUPLPOOL@{\tt ABA\_NONDUPLPOOL}}
that avoids the
multiple storage of the same constraint or variable in the same pool.
The details are explained in Section~\ref{section:nonduplpool}.

\subsection{Visual \CPLUSPLUS\ Compiler}
\index{Visual C++@Visual \CPLUSPLUS}

In addition to the GNU \CPLUSPLUS\ compiler on UNIX operating
systems, \ABACUS\ is now also available on Windows NT in combination with
the Visual \CPLUSPLUS\ compiler. 
Further details for using \ABACUS\ in this new environment can be found
in Section~\ref{chapter:installation}

\subsection{Compiler Preprocessor Flag}
\index{compiler}

In the compilation of an \ABACUS-application the used compiler must be
specified by a preprocessor flag (see Section~\ref{table:compilers}).

\subsection{LP-Solver Preprocessor Flag}

The LP-solvers that are used have to be specified by a preprocessor
flag (see Section~\ref{section:lpsolver}). Also the flags for the
LP-solver Cplex changed.

\subsection{Parameters of Configuration File}

Three new parameters have been added to the configuration file {\tt
.abacus}.

\subsubsection{NBranchingVariableCandidates}

The parameter
{\tt NBranchingVariableCandidates}\index{NBranchingVariableCandidates@{\tt NBranchingVariableCandidates}}
can be
used to control the number of tested branching variables if our
extended strong branching\index{strong branching} concept is used (see
Section~\ref{section:StrongBranching}).

\subsubsection{DefaultLpSolver}

An other new parameter is 
{\tt DefaultLpSolver}\index{DefaultLpSolver@{\tt DefaultLpSolver}} allows
to choose either {\tt Cplex} or {\tt Soplex} as default
LP-solver for the solution of the LP-relaxations.

\subsubsection{SoPlexRepresentation}

Soplex works internally either with column or a row basis. This basis
representation can be selected with the parameter 
{\tt SoPlexRepresentation}\index{SoPlexRepresentation@{\tt
    SoPlexRepresentation}}.
Our tests show that only the row basis works stable in Soplex 1.0.
Further details are explained in Section~\ref{section:parameters}.


\subsection{New Functions}

We implemented several new functions. Some of them might be also
interesting for the users of \ABACUS. For the detailed documentation we
refer to the reference manual.

\begin{itemize}
\item {\tt ABA\_BPRIOQUEUE::getMinKey()}
\item {\tt ABA\_BHEAP::getMinKey()}
\item {\tt bool ABA\_GLOBAL::isInteger(double x)}
\item In addition to the function
      \begin{verbatim}
      void MASTER::initializePools(ABA_BUFFER<ABA_CONSTRAINT*> &constraints,
                                   ABA_BUFFER<ABA_VARIABLE*> &Variables,
                                   int varPoolSize,
                                   int cutPoolSize,
                                   bool dynamicCutPool = false);
      \end{verbatim}
       the function
       \begin{verbatim}
       void MASTER::initializePools(ABA_BUFFER<ABA_CONSTRAINT*> &constraints,
                                    ABA_BUFFER<ABA_CONSTRAINT*> &cuts,
                                    ABA_BUFFER<ABA_VARIABLE*> &Variables,
                                    int varPoolSize,
                                    int cutPoolSize,
                                    bool dynamicCutPool = false);
       \end{verbatim}
       also allows the insertion of some initial cuts into the cut
       pool.

\item Manipulators for setting the width and the precision of
      {\tt ABA\_OSTREAM} have been added that work like the
      corresponding manipulators of the class {\tt ostream}.
      \begin{verbatim}
      ABA_OSTREAM_MANIP_INT setWidth(int w);
      ABA_OSTREAM_MANIP_INT setPrecision(int p);
      \end{verbatim}

\item {\tt  ABA\_OSTREAM::setFormatFlag(fmtflags)}

\item The objective function sense can be changed in the ABA\_LP
  classes with the function
  \begin{verbatim}
  void ABA_LP::sense(const ABA_OPTSENSE &newSense).
  \end{verbatim}

\item The {!=} operator is now available for the class
      {\tt  ABA\_STRING}.
\end{itemize}

\subsection{Miscellaneous}

Besides some bug fixes we made many minor improvements. The most important
ones are listed here.

\begin{itemize}
\item The output for the output levels {\tt SubProblem} and {\tt
 LinearProgram} is formatted in a nicer way.
\item Besides those Cplex parameters that could be directly controlled
      by \ABACUS\ functions, it is now possible to get or to modify
      any Cplex 4.0 and 5.0 parameter with the functions:
      \begin{verbatim}
      int CPLEXIF::CPXgetdblparam(int whichParam, double *value);
      int CPLEXIF::CPXsetdblparam(int whichParam, double value);
      int CPLEXIF::CPXgetintparam(int whichParam, int *value);
      int CPLEXIF::CPXsetintparam(int whichParam, int value);
      \end{verbatim}
\item If a linear program is solved with the barrier method, then
      usually a cross over to an optimal basic solution is
      performed. The value of a variable in the
      optimal solution of the barrier method before 
      the cross over can be obtained with the function
      {\tt double barXVal(int i)}. If this ``pre-cross over'' solution
      is available, can be checked with the function
      {\tt SOLSTAT barXValStatus() const}.

\item The minimal required violation of a constraint or variable
      in a pool separation or pool pricing, respectively, can be
      specified as a parameter of the functions
      {\tt ABA\_SUB::constraintPoolSeparation} and 
      {\tt ABA\_SUB::variablePoolSeparation}. The minimal violation
      is also a parameter of the function  {\tt ABA\_POOL::separate}
      and of redefinitions of this function in derived classes.
\end{itemize}
