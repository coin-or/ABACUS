% E.g., $Id: installation.tex,v 2.8 2007/08/13 13:03:12 baumann Exp $



\chapter{Installation}
\label{chapter:installation}

\section{Obtaining ABACUS}

\ABACUS\ can be obtained from
\begin{center}
{\tt http://www.informatik.uni-koeln.de/abacus/} .
\end{center}
Please note that \ABACUS\ requires a working installation of the Open Solver Interface (Osi) provided by The Computational Infrastructure for Operations Research (COIN-OR) project. Please see Section~\ref{section:lpsolver} for details.
If you have any questions about \ABACUS\, please send a mail to
\begin{center}
{\tt abacus@informatik.uni-koeln.de} .
\end{center}

\section{Platforms}\index{platforms}
\label{section:platforms}

\ABACUS\ is currently available for {\tt linux}.
If you are interested in a version for another platform please contact us directly.

\section{Building \ABACUS\ }
\label{section:lpsolver}
\index{Compiler}
\index{build}
\index{solver}

\ABACUS\ can be compiled with the {\tt GNU-\CPLUSPLUS\ } compilers {\tt g++ 3.3.5 - 4.1.2}.

\ABACUS\ provides a general interface to linear programming solvers.
The current release supports the LP-solvers supported by COIN Osi version 0.96.
Not all of them might be useful in combination with \ABACUS\, though.
Before compiling ABACUS 3.0 make sure that COIN Osi is installed.
For more information on the installation of COIN Osi see
See \htmladdnormallink{{\tt https://projects.coin-or.org/Osi}}{https://projects.coin-or.org/Osi} for details.

Set the paths at the top of the Makefile to the include directories
of COIN Osi and the LP solvers installed on your system.

Settings for different compilers are stored in the directory Make-settings.
Which settings file is used is determined by the variable ABACUS\_MAKE\_SETTINGS.
To compile ABACUS with g++-4.1, for example, do:

make abacus ABACUS\_MAKE\_SETTINGS=linux20-gcc41

To install abacus to a specific location instead of the base directory
set the variables ABACUS\_INSTALL\_LIBDIR and ABACUS\_INSTALL\_HEADERDIR in
the Makefile and run, for example:

make install ABACUS\_MAKE\_SETTINGS=linux20-gcc41

For information on how to produce the documentation, please run:

make


\section{Compiling and Linking}\index{Compiling}\index{Linking}
\label{section:CompilingAndLinking}

For compiling your files using \ABACUS\  add
the {\tt abacus/include} directory either to your include directory
path or specify it explicitly with the {\tt -I} compiler option.
Furthermore, add the include file paths of the LP-solvers you want to use.
The flag for the \CPLUSPLUS\ compiler can be defined at compilation time using 
the {\tt -D} switch of the compiler (e.g., {\tt -DABACUS\_COMPILER\_GCC41})
or specified in the {\tt Makefile}. See table ~\ref{table:compilers} for valid 
settings.
It might be helpful to consult the {\tt Makefile} of the example
included in the \ABACUS\ distribution.

\begin{table}[htp]
\begin{center}
\begin{tabular}{|l|l|}
\hline
compiler preprocessor flags\\
\hline
Linux g++  4.1 & {\tt ABACUS\_COMPILER\_GCC41} or {\tt ABACUS\_COMPILER\_GCC} \\
Linux g++  3.4 & {\tt ABACUS\_COMPILER\_GCC34} \\
Linux g++  3.3 & {\tt ABACUS\_COMPILER\_GCC33} \\
SUN C++ 4.2 & {\tt ABACUS\_COMPILER\_SUN} \\
\hline
\end{tabular}
\caption{compilers}
\label{table:compilers}
\index{ABACUS\_COMPILER\_GCC@{\tt ABACUS\_COMPILER\_GCC}}
\index{ABACUS\_COMPILER\_SUN@{\tt ABACUS\_COMPILER\_SUN}}
\end{center}
\end{table}     

\section{Environment Variables}\index{environment variables}
\label{section:env-variables}
The environment variable 
{\tt ABACUS\_DIR}\index{ABACUS\_DIR@{\tt ABACUS\_DIR}} has to be set to the directory
containing the general configuration file {\tt .abacus}. A master version
of this configuration file is provided in the base directory of the \ABACUS\ distribution.
It is recommended that every user makes a private copy of the file {\tt .abacus} and sets
{\tt ABACUS\_DIR} accordingly.

To set the environment variable to {\tt /home/yourhome}, for example, using the C-shell 
or its relatives, do:
\begin{flushleft}
\tt
setenv ABACUS\_DIR /home/yourhome\\
\end{flushleft}

\noindent
If the Bourne-shell is used do:
\begin{flushleft}
\tt
export ABACUS\_DIR=/home/yourhome\\
\end{flushleft}
Usually it is convenient to add these instructions to the personal {\tt .login}
file.

\section{Contact}\index{contact}\index{bugs}

Feedback from the users is highly appreciated. Please report your
experiences and make your suggestions. Also comments on this user
manual are appreciated. 
Report all problems and suggestions by e-mail to:
\begin{center}
\tt
abacus@informatik.uni-koeln.de
\end{center}
Before reporting a bug, please make sure that it does not come from an
incorrect usage of the programming language \CPLUSPLUS.


\section{Mailing List}\index{mailing list}\index{mailing list}
There is a mailing list available for \ABACUS. 
To subscribe to this service, please register at
\begin{center} 
\htmladdnormallink{{\tt https://lists.uni-koeln.de/mailman/listinfo/abacus-forum}}{https://lists.uni-koeln.de/mailman/listinfo/abacus-forum}
.
\end{center}
