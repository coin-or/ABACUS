% $Id: templates.tex,v 2.0 2007/06/15 11:47:30 sprenger Exp $


\section{Using the ABACUS Templates}
\label{section:UsingTemplates}
\index{templates}

\ABACUS\ also provides several basic data structures as templates. 
For several fundamental types and some \ABACUS\ classes the templates
are instantiated already in the library {\tt libabacus.a}. 
However, if you want to use one of the \ABACUS\ templates
for one of your classes then you have to instantiate the
templates for these classes yourself. 

Moreover, in order to keep the
library small, we instantiated the templates only for those types
which are required in the kernel of the \ABACUS\ system. Therefore,
it can happen that the linker complains about undefined symbols.
In this case you have to instantiate these templates, too.

\ABACUS\ allows implicit or explicit template instantiation. Implicit
template instantiation is the more convenient way. The compiler
automatically instantiates a template when required. Its disadvantage
is that it increases the compile time and (depending on the compiler)
also the size of the generated code. For explicit template
instantiation the templates have to be collected manually in file and
and this file has to be compiled separately. Note, some compilers do
not support explicit template instantiation. Other compilers perform
the explicit template instantiation automatically even if the implicit
instantiation is selected. Currently, we recommend explicit template
instantiation for the GNU-compiler 2.7.x and the SGI compiler and
implicit template instantiation for the GNU-compiler 2.8.x, the SUN compiler
and the MS Visual C++ compiler.

For instance, you want to use an {\tt ABA\_ARRAY} template for your
class {\tt MYCONSTRAINT} and the fundamental type {\tt unsigned int},
for which we have no instantiations in the library {\tt libabacus.a}.
Then you can instantiate explicitly the corresponding templates in a file
{\tt myarray.cc}.

\bigskip
{\tt\obeylines
//
// This is the file myarray.cc.
//

\#include "abacus/array.h"       // the header of the class ABA\_ARRAY
\#include "abacus/array.inc"     // the member functions of the class ABA\_ARRAY

template class ABA\_ARRAY<MYCONSTRAINT>;
template class ABA\_ARRAY<unsigned int>;

// end of file myarray.cc
}
\bigskip

\noindent
The file {\tt myarray.cc} should be compiled and linked together with
your files and the library {\tt libabacus.a}. In the file in which 
you are using the array templates only the file {\tt array.h} should
be included.

For more information on templates we refer to the documentation of 
the \htmladdnormallinkfoot{templates for the GNU compiler}
        {http://funnelweb.utcc.utk.edu/\~{}harp/gnu/gcc-2.7.0/gcc\_98.html\#SEC101}.
We prefer the method using the g++ compiler flag 
{\tt -fno-implicit-templates}.\index{-fno-implicit-templates@{\tt -fno-implicit-templates}}
