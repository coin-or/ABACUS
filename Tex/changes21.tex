\section{New Features of ABACUS 2.2}
\label{section:new22}


Version 2.2 includes a new interface to the Lp-Solver
Xpress and Cplex 6.0 and it provides enhaced functionality
for parameter handling.
Moreover, the library is now available for different native compilers.
It can be configured for any combination of supported LP-Solvers by the
user.
\ABACUS\ now intensively uses inline functions to improve performance.

\subsection{Lp-Solver Xpress}

In addition to the LP-Solvers Cplex and Soplex ABACUS now also provides
an interface to the LP-Solver Xpress-MP Version 10.
The Xpress libraries {\tt libxosl.a} and {\tt libmp-opt.a} have both to
be linked.

Xpress-MP is a commercial product by Dash Associates. You find further
information about Xpress at
\htmladdnormallink{{\tt http://www.dash.co.uk/}}{http://www.dash.co.uk/}.

\subsection{Lp-Solver Cplex}

Cplex 6.0 is now supported.

In addition, a new parameter {\tt CplexHoldEnvironment} is introduced.
If this parameter is true, then the Cplex environment 
is held open during the branch-and-cut optimization.
This reserves a Cplex license for the complete time of
optimization.

\subsection{Lp-Methods}

\begin{sloppypar}
The solution method for linear programs
{\tt LP::Barrier} is replaced by the methods
{\tt LP::BarrierAndCrossover} and {\tt LP::BarrierNoCrossover}.
\end{sloppypar}

\subsection{New Compilers}

New supported compilers are the {\tt GNU C++ compiler gcc 2.8} and the
{\tt Sun WorkShop Compiler C++ 4.2}. We now provide 32 and 64 bit versions
of the \ABACUS\ library compiled with the {\tt SGI MIPSpro 7.2 C++ compiler}.

\subsection{Library Architectures}

The \ABACUS\ library is provided for different combinations of {\tt hardware},
{\tt operating systems} and {\tt compilers}.
These combinations are identified by an {\tt <arch>} name.
Some architectures are shown in table \ref{table:arch}.
\begin{table}[htp]
\begin{center}
\begin{tabular}{|l|l|l|l|}
\hline
Hardware & Operating System & Compiler & <arch>\\
\hline
SUN SPARC & SUN-OS 4.1.3 & GNU C++ Compiler 2.8.1 & sunos-gcc28 \\
SUN SPARC & SUN-OS 5.6 & GNU C++ Compiler 2.8.1 & solaris-gcc28 \\
SUN SPARC & SUN-OS 5.6 & GNU C++ Compiler 2.7.2 & solaris-gcc27 \\
SUN SPARC & SUN-OS 5.6 & SUN WorkShop Compiler C++ 4.2 & solaris-CC \\
IBM RS6000 & AIX 4.1.5 & GNU C++ Compiler 2.8.1 & aix4-gcc28 \\
IBM RS6000 & AIX 4.1.5 & GNU C++ Compiler 2.7.2 & aix4-gcc27 \\
DEC ALPHA & OSF 3.2 & GNU C++ Compiler 2.8.1 & osf-gcc28 \\
DEC ALPHA & OSF 3.2 & GNU C++ Compiler 2.7.2 & osf-gcc27 \\
SILICON GRAPHICS & Irix 6.2 & GNU C++ Compiler 2.7.2 & irix6-gcc27 \\
SILICON GRAPHICS & Irix 6.2 & MIPSpro 7.2 C++ compiler 32 Bit,  & irix6-CCn32 \\
& & mips4 & \\
SILICON GRAPHICS & Irix 6.2 & MIPSpro 7.2 C++ compiler 64 Bit, & irix6-CC64 \\
& & mips4 & \\
HP 9000 & HP-UX 10.20 & GNU C++ Compiler 2.8.1 & hpux10-gcc28 \\
PC & Linux 2.0.27 & GNU C++ Compiler 2.8.1 & linux20-gcc28 \\
PC & Linux 2.0.27 & GNU C++ Compiler 2.7.2 & linux20-gcc27 \\
PC & Windows NT & MS Visual C++ 5.0 & winnt \\
\hline
\end{tabular}
\caption{Architecture names.}
\label{table:arch}
\end{center}
\end{table}

\subsection{Library Creation by the User}

The library archive file {\tt abacus-<version>-<arch>.tar.gz} contains
the basic \ABACUS\ library and libraries for each supported
{\tt Interface} to an LP-Solver. Currently supported Interfaces are
shown in table \ref{table:interfaces}.

\begin{table}[htp]
\begin{center}
\begin{tabular}{|l|l|}
\hline
Interface name & LP-Solver\\
\hline
cplex22 &                       Cplex 2.2 \\
cplex30 &                       Cplex 3.0 \\
cplex40 &                       Cplex 4.0 \\
cplex50 &                       Cplex 5.0 \\
cplex60 &                       Cplex 6.0 \\
soplex  &                       Soplex 1.0 \\
xpress  &                       Xpress-MP 10 \\
\hline
\end{tabular}
\caption{Interface names.}
\label{table:interfaces}
\end{center}
\end{table}

You can create \ABACUS\ libraries for any combination of supported
LP-Solvers by yourself.
Downloaded and unpack the library ditribution archive with the right
<arch> in the installation root directiry (e.g. /usr/local/abacus).
A directory {\tt abacus-<version/lib/<arch>/stuff} is created which
contains all required files to build specific \ABACUS\ libraries.
Then create LP-Solver specific ABACUS libraries by using the command
{\tt make-lib} in the directory {\tt lib/<arch>} for any desired
combination of different LP-solvers.

For example if you want to have \ABACUS\ libraries for Solaris compiled
with gcc 2.8 download the file {\tt abacus-2.2-solaris-gcc28.tar.gz}.
\begin{verbatim}
  gunzip abacus-2.2-solaris-gcc28.tar.gz
  tar xvf abacus-2.2-solaris-gcc28.tar
\end{verbatim}
To create libraries with interfaces for Cplex 6.0, Soplex, Xpress and
all three together type
\begin{verbatim}
  cd abacus-2.2/lib/solaris-gcc28
  make-lib cplex60
  make-lib soplex
  make-lib xpress
  make-lib cplex60-soplex-xpress
\end{verbatim}
The {\tt make-lib <interfaces>} command creates the file

\centerline{{\tt abacus-<version>/lib/solaris-gcc28/libabacus-<interfaces>.a},}

 where {\tt <interfaces>} is an
interface string or a combination of interface strings concatenated by the
character {\tt -}.

\subsection{New or Changed Preprocessor Flags}

A list of preprocessor flags with new or changed meaning follows:

\begin{center}
\begin{tabular}{|l|l|}
\hline
Preprocessor Flag & Meaning\\
\hline
ABACUS\_COMPILER\_GCC28     &       GNU C++ compiler 2.8 \\
ABACUS\_COMPILER\_GCC27     &       GNU C++ compiler 2.7 \\
ABACUS\_COMPILER\_GCC       &       defaults to ABACUS\_COMPILER\_GCC28 \\
ABACUS\_COMPILER\_SUN       &       SUN WorkShop C++ Compiler 4.2 \\
\hline
ABACUS\_EXPLICIT\_TEMPLATES &       no longer needed \\
ABACUS\_CPP\_MATH           &       no longer needed \\
ABACUS\_SYS\_xxxxxx         &       no longer needed \\
ABACUS\_LP\_SOPLEX          &       no longer needed \\
ABACUS\_LP\_CPLEXxx         &       needed only if lpmastercplex.h
                                  or cplexif.h is included. \\
\hline
\end{tabular}
\end{center}

See the updated Makefile of the TSP example in
{\tt abacus-2.2/example/Makefile} for a description of the
valid compiler and linker flags. This file also explains how to link
an application with more than one LP solver.


\subsection{Templates}
\index{templates}

It is no longer needed to include template definition files (*.inc). These
files are now automatically included by the coresponding header files (*.h).

If you are using gcc 2.8 no special flags for template instatiation need to be
defined. If you are using gcc 2.7 we recommended to define the compilerflag
{\tt -fno-implicit-templates} and to manually instantiate the templates
which are needed, but not contained in the ABACUS library as described in
section \ref{section:UsingTemplates}.


\subsection{New LP Master Classes}

There is a new abstract class {\tt ABA\_LPMASTER} and subclasses
{\tt ABA\_LPMASTERCPLEX}, {\tt ABA\_LPMASTERSOPLEX} and {\tt ABA\_LPMASTERXPRESS}.
These classes handle LP solver
specific parameters and global data. As a consequence some LP solver specific
functions which were located in ABA\_MASTER are now located in one of these
classes. If you are using such a function you have to change your code as
shown in the example below:
\begin{verbatim}
    master->cplexOutputLevel(level);
\end{verbatim}
should be changed to
\begin{verbatim}
    ABA_LPMASTERCPLEX *cplexMaster = master->lpMasterCplex();
    lpMasterCplex->cplexOutputLevel(level);
\end{verbatim}
or simply
\begin{verbatim}
    master->lpMasterCplex()->cplexOutputLevel(level);
\end{verbatim}

\begin{sloppypar}
The corresponding header files are {\tt abacus/lpmastercplex.h}, 
{\tt abacus/lpmastersoplex.h}, and {\tt abacus/lpmasterxpress.h}.
\end{sloppypar}

\subsection{HTML Documentation}

In the HTML version of the Reference Manual 
(Section \ref{chapter:ReferenceManual}), we added links in
the declaration part of the class which point to other classes
and to the descriptions of the class members.


\subsection{Parameter Handling}
 
The system parameter table and the functions for handling parameters
moved from class {\tt ABA\_MASTER} to its base class {\tt ABA\_GLOBAL}.
Now, it is possible to use the parameter concept of \ABACUS\ even without
generating an object of {\tt ABA\_MASTER}. (This might be useful when
writing some experimental code using the Tools and Templates of
\ABACUS, but not writting an complete \bac-application.)

In addition to the overloaded functions {\tt ABA\_GLOBAL::getParameter()},
 we now provide the overloaded functions {\tt ABA\_GLOBAL::assignParameter()}
and {\tt ABA\_GLOBAL::findParameter()} with enhanched functionality.
The
new functions test for the existence of a parameter in the
table, compare the current setting with feasible settings and allow
for termination of the program if a 
required paramter is not found, or if it is found but if its
setting is not feasible.

Moreover,  a \bac-optimization can be started without reading
the parameter file {\tt .abacus}.

See section~\ref{section:parameterHandling} for further details on
using parameters.
 

\subsection{Name changings}

This version contains some changings of names that seemed
reasonable to us. Most changings were guided by the principle
that we want to have the feasible values of the 
\ABACUS\ parameters coinciding with the 
enumerators of the corresponding enumeration type.
(As all enumerators in one class have to be different, an 
exception to that rule is the parameter value {\ttfamily None}
which is feasible for different paramaters.)

In addition, we changed in general in {\ttfamily SoPlex} the upper 
{\ttfamily P} to a lower one.

\smallskip

Table~\ref{table:namechangings21} summarizes the changings. We provide
Perl scripts for performing the changings on your \ABACUS\ application.
For a parameter file, use 
\begin{verbatim}
upd-parameter-2.2 <parameter-file>
\end{verbatim}
and apply
\begin{verbatim}
upd-sources-2.2 <code-filse>
\end{verbatim}
to your C++ code files.

\begin{table}[htb]
\begin{center}
\begin{tabular}{c||c|c}
Location&Old &New\\
\hline
Parameter {\tt EnumerationStrategy}&{\tt Best}&{\tt BestFirst}\\
Parameter {\tt EnumerationStrategy}&{\tt Depth}&{\tt DepthFirst}\\
Parameter {\tt EnumerationStrategy}&{\tt Breadth}&{\tt BreadthFirst}\\
{\tt ABA\_MASTER::PRIMALBOUNDMODE}&{\tt OptimalPrimalBound}&{\tt Optimum}\\
{\tt ABA\_MASTER::PRIMALBOUNDMODE}&{\tt OptimalOnePrimalBound}&{\tt OptimumOne}\\
{\tt ABA\_MASTER::VBCMODE}&{\tt None}&{\tt NoVbcLog}\\
various &{\tt SoPlex}&{\tt Soplex}\\
{\tt ABA\_LP::METHOD}&{\tt Barrier}&{\tt BarrierAndCrossover}\\
\end{tabular}
\caption{Name changings.}
\label{table:namechangings21}
\end{center}
\end{table}

\section{New Features of ABACUS 2.1}
\label{section:new21}

In version 2.1 we added a few new features, fixed some bugs, and
improved the performance of some functions.
 

 
\subsection{Elimination of Constraints and Variables}

So far a constraint or variable was eliminated from the set of active
items as soon as the criterion for elimination hold. Now the number of
iterations the criterion must hold until the elimination is performed
can be specified in the configuration file {\tt .abacus} (see
Section~\ref{section:parameters}). 
 
\subsection{Cplex 5.0}

Cplex 5.0 is now supported by \ABACUS.

\subsection{Templates}

In addition to the explicit instantiation of templates, \ABACUS\ now
also supports the implicit instantiation (see
Section~\ref{section:UsingTemplates}). 


\subsection{Bug Fixes}


\subsubsection{Constraint and Variable Selection}

The selection of constraints and variables with highest rank from the 
buffers of generated constraints and variables is now performed
correctly again.

\subsubsection{Variable Generation}

We have tested the dymanic variable generation of \ABACUS\ more
intensively and could fix some so far unknown bugs.

