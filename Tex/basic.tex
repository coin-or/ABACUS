% $Id: basic.tex,v 2.2 2007/07/24 16:02:27 baumann Exp $

\chapter{Using ABACUS}
\label{chapter:UsingAbacus}

Section~\ref{section:UsingAbacusBasics} provides the basic guidelines 
how a new application can be attacked with the help of \ABACUS. While
this section describes the first steps a user should follow, we discuss
in Section~\ref{section:UsingAbacusAdvanced}
advanced features, in particular how default strategies
can be modified according to problem specific requirements.

We strongly encourage to study this chapter together with the example
of the \ABACUS\ distribution. In this example all concepts of
Section~\ref{section:UsingAbacusBasics} and several features of 
Section~\ref{section:UsingAbacusAdvanced} can be found.

In the following sections we also present pieces of \CPLUSPLUS\ code.
When we discuss variables that are of the type ``pointer to some type'',
then we usually omit for convenience of presentation the ``pointer to''
and the operator $*$ if there is no danger of confusion.
For instance, given the variable
\begin{verbatim}
  ABA_ARRAY<ABA_CONSTRAINT*> *constraints;
\end{verbatim}
we also say ``the constraints are stored in the array {\tt constraints}''
instead of ``the pointer to constraints are stored in the array
{\tt *constraints}''.


In order to simplify the use \ABACUS\ we are using the following style
for the names of classes, functions, variables, and enumerations.
\index{naming style}

\begin{itemize}
\item  Names of classes (e.g., {\tt class ABA\_COLUMN}), 
       names of enumerations (e.g., {\tt enum VARELIMMODE}), 
       and character strings associated with an enumeration
       (e.g., {\tt const char* VARELIMMODE\_[]})       
       are written with upper case letters.

\item  Members of enumerations begin with an upper case letter,
       e.g., {\tt enum STATUS\{Fixed, Set\}}.

\item All other names (functions, objects, variables, function
         arguments) start with a lower case letter
         (e.g., {\tt optimize()}).

\item We use upper case letters within all names to increase the
      readability (e.g., {\tt generateSon()}).

\item Names of data members of classes end with an underscore
          such that they can be easily distinguished from local
          variables of member functions.

\item We do not refrain from using a long name if it helps
      expressing the concepts behind the name.
\end{itemize}


\section{Basics}
\label{section:UsingAbacusBasics}

In this section we explain how our framework is used for the implementation
of a new application. 
This section should provide only the guidelines for the first steps
of an implementation, for
details we refer to Section~\ref{section:UsingAbacusAdvanced}
and to the documentation in the reference manual.

If we want to use \ABACUS\ for a new application we have to derive problem
specific classes from some base classes. Usually, only four base classes
of \ABACUS\ are involved: {\tt ABA\_VARIABLE}, {\tt ABA\_CONSTRAINT}, 
{\tt ABA\_MASTER}, and {\tt SUBPROBLEM}. For some applications it is
even possible that the classes {\tt ABA\_VARIABLE} and/or {\tt ABA\_CONSTRAINT}
are not included in the derivation process if those concepts
provided already by \ABACUS\ are sufficient. By the definition of some
pure virtual functions of the base classes in the derived classes and
the redefinition of some virtual functions a problem specific algorithm
can be composed. Figure~\ref{figure:derive} shows how the problem
specific classes {\tt MYMASTER}, {\tt MYSUB}, {\tt MYVARIABLE}, and
{\tt MYCONSTRAINT} are embedded in the inheritance graph of \ABACUS.

\begin{figure}
\centerline{
  \psfig{figure=figures/derive.ps,width=150truemm}
}
\caption{Embedding problem specific classes in \ABACUS.}
\label{figure:derive}
\end{figure}


Throughout this section we only use the default pool concept of \ABACUS,
i.e., we have one pool for static constraints, one pool for
dynamically generated cutting planes,
and one pool for variables. We will outline how an application specific
pool concept can be implemented in Section~\ref{section:otherPools}.

\subsection{Constraints and Variables}
\index{constraint}
\index{variable}

The first step in the implementation of a new application is the analysis
of its variable and constraint structure. We require at least one
constraint class derived from the 
class~{\tt ABA\_CONSTRAINT}\index{ABA\_CONSTRAINT@{\tt ABA\_CONSTRAINT}} and at least one
variable class derived from the class {\tt ABA\_VARIABLE}\index{ABA\_VARIABLE@{\tt ABA\_VARIABLE}}. 
The used variable
and constraint classes have to match such that a row or a column
of the constraint matrix of an LP-relaxation can be generated.

We derive from the class {\tt ABA\_VARIABLE} the class {\tt MYVARIABLE} storing
the attributes specific to the variables of our application, e.g.,
its number, or the tail and the head of the associated edge of a graph.

Then we derive the class {\tt MYCONSTRAINT} from the class
{\tt ABA\_CONSTRAINT}
\begin{verbatim}
class MYCONSTRAINT : public ABA_CONSTRAINT {
  public:
    virtual double coeff(ABA_VARIABLE *v);
};
\end{verbatim}
The function {\tt  ABA\_CONSTRAINT::coeff(ABA\_VARIABLE *v)}
is a pure virtual function. Hence, we define it in the class
{\tt MYCONSTRAINT}. It returns the coefficient of variable {\tt v}
in the constraint.
Usually, we need in an implementation of the function
{\tt coeff(ABA\_VARIABLE *v)}
access to the application specific attributes of the variable {\tt v}.
Therefore, we have to cast {\tt v} to a pointer to an object of the 
class {\tt MYVARIABLE} for the computation of the coefficient of {\tt v}.
Such that this cast can be performed safely, the variables and constraints
used within an application have to be compatible. If run time type
information (RTTI) is supported
on your system, these casts can be performed safely.

The function {\tt coeff()} is used within the framework when the row
format of a constraint is computed, e.g., when the linear program
is set up, or a constraint is added to the linear program.
When the column associated with a variable is generated, then the virtual
member function {\tt coeff()} of the class {\tt ABA\_VARIABLE} is used, which
is in contrast to the function~{\tt coeff()} of the class {\tt ABA\_CONSTRAINT}
not an abstract function:
\begin{verbatim}
  double ABA_VARIABLE::coeff(ABA_CONSTRAINT *con)
  {
    return con->coeff(this);
  }
\end{verbatim}
This method of defining the coefficients of the constraint matrix via
the constraints of the matrix originates from cutting plane algorithms.
Whereas in a column generation algorithm we usually have a different view on the 
problem, i.e., the coefficients of the constraint matrix are defined
with the help of the variables. In this case, it is appropriate to define
the function {\tt MYCONSTRAINT::coeff(ABA\_VARIABLE *v)} analogously to the function 
{\tt ABA\_VARIABLE::coeff(ABA\_CONSTRAINT *v)} and to define the the function
{\tt MYVARIABLE::coeff(ABA\_CONSTRAINT *v)}.


\ABACUS\ provides two constraint/variable pairs in its application independent
kernel.
The most simple one is where each variable is identified by
an integer number (class {\tt ABA\_NUMVAR}\index{ABA\_NUMVAR@{\tt ABA\_NUMVAR}}) and each 
constraint is represented
by its nonzero coefficients and the corresponding number of the variables
(class {\tt ABA\_ROWCON}\index{ABA\_ROWCON@{\tt ABA\_ROWCON}}). We use this constraint/variable pair
for general \mip s.

The constraint/variable pair 
{\tt ABA\_NUMCON}\index{ABA\_NUMCON@{\tt ABA\_NUMCON}}/{\tt ABA\_COLVAR}\index{ABA\_COLVAR@{\tt ABA\_COLVAR}} is dual to the previous
one. Here the constraints are given by an integer number, but we store
the nonzero coefficients and the corresponding row numbers for each variable.
Therefore, this constraint/variable pair is useful for column generation algorithms.

\ABACUS\ is not restricted to a single constraint/variable pair within
one application. There can be an arbitrary number of constraint and variable
classes. It is only required that the coefficients of the constraint
matrix can be safely computed for each constraint/variable pair.

\subsection{The Master}
\index{master}

There are two main reasons why we require a problem specific master 
of the optimization. The first reason is that we have to embed
problem specific data members like the problem formulation.
The second reason is the initialization of the first subproblem, i.e., the
root node of the \bab\ tree has to be initialized with a subproblem
of the class {\tt MYSUB}. Therefore, a problem specific master
has to be derived from the class {\tt ABA\_MASTER}\index{ABA\_MASTER@{\tt ABA\_MASTER}}:
\begin{verbatim}
class MYMASTER : public ABA_MASTER {};
\end{verbatim}

\subsubsection{The Constructor}

Usually, the input data is read from a file by the constructor or they are
specified by the arguments of the constructor. 

From the constructor of the class {\tt MYMASTER} the constructor of the base 
class {\tt ABA\_MASTER} must be called:
\begin{verbatim}
  ABA_MASTER(const char *problemName, bool cutting, bool pricing,
             ABA_OPTSENSE::SENSE optSense = ABA_OPTSENSE::Unknown,
             double eps = 1.0e-4, double machineEps = 1.0e-7,
             double infinity = 1.0e32);
\end{verbatim}\index{ABA\_MASTER@{\tt ABA\_MASTER}!Constructor@{\tt Constructor}}
Whereas the first three arguments are mandatory, the other ones are
optional.

\begin{center}
\begin{tabular}[b]{lp{0.75\textwidth}}
  {\tt problemName}&The name of the problem being solved.\\
  {\tt cutting}    &If {\tt true}, then cutting planes are generated.\\
  {\tt pricing}    &If {\tt true}, then inactive variables are priced out.\\
  {\tt optSense}   &The sense of the optimization.\\
  {\tt eps}        &A zero-tolerance used within all member functions of
                    objects that have a pointer to this global object.\\
  {\tt machineEps} &Another zero tolerance to compare a
                    value of a floating point variable with 0. This value
                    is usually less than {\tt eps}, because 
                    {\tt eps} includes
                    some ``safety'' tolerance, e.g., to test if a constraint
                    is violated.\\
  {\tt infinity}   &All floating point numbers greater than {\tt infinity} are
                    regarded as ``infinitely big''.
		    Please note that this value might be different from 
		    the value the LP-solver uses internally.
		    You should make sure that the value used here is always 
		    greater than or equal to the value used by the solver.\\
\end{tabular}
\end{center}

An optional argument of the constructor of the class {\tt ABA\_MASTER}
is the sense of the optimization. For some problems (e.g., the
\bcsp\ or the \tsp) the sense of the optimization is already 
known when this constructor is called. For other problems
(e.g, the \mip) the sense of the optimization is determined later when
the input data is read in the constructor of the specific application.
In this case, the sense of the optimization has to be initialized
explicitly
before the optimization is started with the function {\tt optimize()}.

The following example of a constructor for the class {\tt MYMASTER} sets up
the master for a \bac\ algorithm and initializes the optimization sense
explicitly as it is read from the input file.
\begin{verbatim}
  MYMASTER::MYMASTER(const char *problemName) :
    ABA_MASTER(problemName, true, false),
  {
    // read the data from the file problemName
    if (/* problemName is a minimization problem*/)
      initializeOptSense(ABA_OPTSENSE::Min);
    else
      initializeOptSense(ABA_OPTSENSE::Max);
  }
\end{verbatim}

\subsubsection{Initialization of the Constraints and Variables}

The constraints and variables that are not generated dynamically, e.g., the
degree constraints of the \tsp\ or the constraints and variables of the
problem formulation of a general \mip, have to
be set up and inserted in pools in a member function of 
the class {\tt MYMASTER}. These initializations
can be also performed in the constructor, but we recommend to use
the virtual dummy function 
{\tt initializeOptimization()}\index{ABA\_MASTER@{\tt ABA\_MASTER}!initializeOptimization@{\tt initializeOptimization}}
for this purpose,
which is called after the optimization
is started with the function {\tt optimize()}.

By default, \ABACUS\ provides three different pools\index{pool}: 
one for variables
and two for constraints. The first constraint pool stores the constraints
that are not dynamically generated and with which the first LP-relaxation
of the first subproblem is initialized. The second constraint pool is
empty at the beginning and is filled up with dynamically generated cutting
planes. In general, \ABACUS\ provides a more flexible pool concept to
which we will come back later, but for many applications the default 
pools are sufficient.

After the initial variables and constraints are generated they have to be 
inserted into the default pools by calling the function
\begin{verbatim}
  virtual void initializePools(ABA_BUFFER<ABA_CONSTRAINT*> &constraints,
                               ABA_BUFFER<ABA_VARIABLE*>   &variables,
                               int                          varPoolSize,
                               int                          cutPoolSize,
                               bool                         dynamicCutPool = false);
\end{verbatim}
\index{ABA\_MASTER@{\tt ABA\_MASTER}!initializePools@{\tt initializePools}}
Here, {\tt constraints} are the initial constraints, {\tt variables} are
the initial variables, {\tt varPoolSize} is the initial size of the variable pool,
and {\tt cutPoolSize} is the initial size of the cutting plane pool. The 
size of the variable pool is always dynamic, i.e., this pool is increased
if required. By default, the size of the cutting plane pool is fixed,
but it becomes dynamic if the argument {\tt dynamicCutPool} is {\tt true}.

There is second version of the function |initializePools()| that
allows the insertion of an initial set of cutting planes into the cut
pool\index{pool!initial cutting planes}.

The function {\tt initializeOptimization()} can be also used to determine
a feasible solution by a heuristic such that the primal bound can
be initialized.

Hence, the function {\tt initializeOptimization()} could look as follows
under the assumption that the functions {\tt nVar()} and {\tt nCon()}
are defined in the class {\tt MYMASTER} and return the number of variables
and the number of the constraints, respectively.  In the example
we initialize the size of the cut pool with {\tt 2*nCon()}.
As the arguments
of the constructors of the classes {\tt MYVARIABLE} and {\tt MYCONSTRAINT} are
problem specific we replace them by ``\dots''.

After the pools are set up the primal bound is initialized with
the value of a feasible solution returned by the function {\tt myHeuristic()}.
While the initialization of the pools is mandatory the initialization
of the primal bound is optional.
\begin{verbatim}
  void MYMASTER::initializeOptimization()
  {
    ABA_BUFFER<ABA_VARIABLE*> variables(this, nVar());
    for (int i = 0; i < nVar(); i++)
      variables.push(new MYVARIABLE(...));
    ABA_BUFFER<ABA_CONSTRAINT*> constraints(this, nCon());
    for (i = 0; i < nCon(); i++)
      constraints.push(new MYCONSTRAINT(...));
    initializePools(constraints, variables, nVar(), 2*nCon());
    primalBound(myHeuristic());
  }
\end{verbatim}

\subsubsection{The First Subproblem}

The root of the \bab\ tree has to be initialized with an object of the
problem specific subproblem class {\tt MYSUB}, which is derived from the
class {\tt ABA\_SUB}\index{ABA\_SUB@{\tt ABA\_SUB}}. 
This initialization must be performed by a definition
of the pure virtual function {\tt firstSub()}, which returns a pointer
to the first subproblem. In the following example we assume that
the constructor of the class {\tt MYSUB} for the root node of the
enumeration tree has only a pointer to the associated master as argument.
\begin{verbatim}
  ABA_SUB *MYMASTER::firstSub()
  {
    return new MYSUB(this);
  }
\end{verbatim}\index{ABA\_SUB@{\tt ABA\_SUB}!firstSub@{\tt firstSub}}

\subsection{The Subproblem}

Finally, we have to derive a problem specific subproblem from the class
{\tt ABA\_SUB}\index{ABA\_SUB@{\tt ABA\_SUB}}:
\begin{verbatim}
  class MYSUB : public ABA_SUB {};
\end{verbatim}
Besides the constructors only two pure virtual functions of the base
class {\tt ABA\_SUB} have to be defined, which check if a solution of the 
LP-relaxation is a feasible solution of the \mip, and generate the
sons after a branching step, respectively. Moreover, the main functionality
of the problem specific subproblem is to enhance the \bab\ algorithm
with dynamic variable and constraint generation and sophisticated
primal heuristics.

\subsubsection{The Constructors}
\index{ABA\_SUB@{\tt ABA\_SUB}!Constructor@{\tt Constructor}}

The class {\tt ABA\_SUB} has two different constructors: one for the root node
of the optimization and one for all other subproblems of the optimization.
This differentiation is required as the constraint and variable set
of the root node can be initialized explicitly, whereas for the other nodes
this data is copied from the father node and possibly modified by
a branching rule. Therefore, we also have 
to implement these two constructors for the class {\tt MYSUB}.

The root node constructor for the class {\tt ABA\_SUB} must be called
from the root node constructor of the class {\tt MYSUB}.
\begin{verbatim}
  ABA_SUB(ABA_MASTER *master,
          double conRes, double varRes, double nnzRes,
          bool relativeRes = true,
          ABA_BUFFER<ABA_POOLSLOT<ABA_CONSTRAINT, ABA_VARIABLE> *> *constraints = 0,
          ABA_BUFFER<ABA_POOLSLOT<ABA_VARIABLE, ABA_CONSTRAINT> *> *variables = 0);
\end{verbatim}

\begin{center}
\begin{tabular}{lp{0.8\textwidth}}
  {\tt master}       &A pointer to the corresponding master of the
                      optimization.\\
  {\tt conRes}       &The additional memory allocated for constraints.\\
  {\tt varRes}       &The additional memory allocated for variables.\\
  {\tt nnzRes}       &The additional memory allocated for nonzero
                      elements of the constraint matrix.\\
  {\tt relativeRes}  &If this argument is {\tt true}, then reserve space
                      for variables, constraints, and nonzeros of
                      the previous three arguments is given in
                      percent of the original numbers. Otherwise,
                      the numbers are interpreted as absolute values
                      (casted to integer).\\
  {\tt constraints}  &The pool slots of the initial constraints.
                      If the value is 0, then all constraints
                      of the default constraint pool are taken.\\
  {\tt variables}    &The pool slots of the initial variables.
                      If the value is 0, then all variables of the
                      default variable pool are taken.\\

\end{tabular}
\end{center}

\noindent
The values of 
the arguments {\tt conRes}, {\tt varRes}, and {\tt nnzRes} should only be
good estimations. An underestimation does not cause a run time error,
because space is reallocated internally as required. However,
many reallocations decrease the performance. 
An overestimation only wastes memory.

In the following implementation of a constructor for the root node
we do not specify additional memory for variables, because we suppose
that no variables are generated dynamically. We accept the default settings of the last three arguments, as this
is normally a good choice for many applications.
\begin{verbatim}
  MYSUB::MYSUB(MYMASTER *master) :
    ABA_SUB(master, 50.0, 0.0, 100.0)
  { }
\end{verbatim}
While there are some alternatives for the implementation of the
root node for the application, the constructor of non-root nodes
has usually the same form for all applications, but might be
augmented with some problem specific initializations.
\begin{verbatim}
  MYSUB::MYSUB(ABA_MASTER *master, ABA_SUB *father, ABA_BRANCHRULE *branchRule) :
    ABA_SUB(master, father, branchRule) 
  {}
\end{verbatim}

\begin{center}
\begin{tabular}{lp{0.75\textwidth}}
  {\tt master}     &A pointer to the corresponding master of the
                    optimization.\\
  {\tt father}     &A pointer to the father in the enumeration tree.\\
  {\tt branchRule} &The rule defining the subspace of the
                    solution space associated with this node. More
                    information about branching rules can be found
                    in Section~\ref{section:otherBranching}. As long
                    as you are using only the default branching on
                    variables you do not have to know anything about
                    the class {\tt ABA\_BRANCHRULE}.\\
\end{tabular}
\end{center}

The root node constructor for the class {\tt MYSUB} has to be called
from the function {\tt firstSub()} of the class {\tt MYMASTER}. The 
constructor for non-root nodes has to be called in the function
{\tt generateSon()} of the class {\tt MYSUB}.

\subsubsection{The Feasibility Check}
\index{ABA\_SUB@{\tt ABA\_SUB}!feasible@{\tt feasible}}

After the  LP-relaxation is solved we have to check if its optimum solution
is a feasible solution of our optimization problem. Therefore, we
have to define the pure virtual function {\tt feasible()} in the class
{\tt MYSUB}, which should return {\tt true} if the LP-solution is
a feasible solution of the optimization problem, and {\tt false} otherwise:
\begin{verbatim}
  bool MYSUB::feasible()
  {}
\end{verbatim}
\noindent
If all constraints of the integer programming formulation are present in
the LP-relaxation, then the LP-solution is feasible if all discrete variables
have integer values. This check can be performed by calling the member function
{integerFeasible()} of the base class {\tt ABA\_SUB}:
\begin{verbatim}
  bool MYSUB::feasible()
  {
    return integerFeasible();
  }
\end{verbatim}
If the LP-solution is feasible and its value is better than the primal
bound, then \ABACUS\ automatically updates the primal bound. However,
the update of the solution itself is problem specific, i.e., this update
has to be performed within the function {\tt feasible()}.

\subsubsection{The Generation of the Sons}
\index{ABA\_SUB@{\tt ABA\_SUB}!generateSon@{\tt generateSon}}

Like the pure virtual function {\tt firstSub()} of the class {\tt ABA\_MASTER},
which generates the root node of the \bab\ tree, we need a function
generating a son of a subproblem. This function is required as the
nodes of the \bab\ tree have to be identified with a problem specific
subproblem of the class {\tt MYSUB}.
This is performed by the pure 
virtual function {\tt generateSon()}, which calls the constructor for
a non-root node of the class~{\tt MYSUB} and returns a pointer to the
newly generated subproblem. If the constructor for non-root nodes
of the class {\tt MYSUB} has the same arguments as the corresponding
constructor of the base class {\tt ABA\_SUB}, then the function
{\tt generateSon()} can have the following form:
\begin{verbatim}
  ABA_SUB *MYSUB::generateSon(ABA_BRANCHRULE *rule)
  {
    return new MYSUB(master_, this, rule);
  }
\end{verbatim}
This function is automatically called during a branching process. If the
already built-in branching strategies are used, we
do not have to care about the generation of the branching rule {\tt rule}.
How other branching strategies can be implemented is presented in 
Section~\ref{section:otherBranching}.

\subsubsection{A Branch-and-Bound Algorithm}
\noindent
The two constructors, the function {\tt feasible()}, and the function
{\tt generateSon()} must be implemented for the subproblem class of
every application. As soon as these functions are available, a 
\bab\ algorithm can be performed. All other functions of the
class {\tt MYSUB} that we are going to explain now, are optional
in order to improve the performance of the implementation.

\subsubsection{The Separation}
\index{ABA\_SUB@{\tt ABA\_SUB}!separate@{\tt separate}}
\index{separation}

Problem specific cutting planes can be generated by redefining the
virtual dummy function {\tt separate()}. In this case, also the argument
{\tt cutting} in the constructor of the class~{\tt ABA\_MASTER} should receive
the value {\tt true}, otherwise the separation is skipped.
The first step is the redefinition of the function {\tt separate()} of
the base class {\tt ABA\_SUB}.
\begin{verbatim}
  int MYSUB::separate()
  { }
\end{verbatim}
The function {\tt separate()} returns the number of generated constraints.

We distinguish between the
separation from scratch and the separation from a constraint pool.
Newly generated constraints
have to be added by the 
function {\tt addCons()}\index{ABA\_SUB@{\tt ABA\_SUB}!addCons@{\tt addCons}} to the
buffer of the class {\tt ABA\_SUB}, which returns the number
of added constraints.
Constraints generated in earlier iterations that have been become
inactive in the meantime might be still contained in the cut pool.
These constraints can be regenerated by calling the 
function~{\tt constraintPoolSeparation()}, 
which adds the constraints to the buffer without
an explicit call of the function {\tt addCons()}.

A very simple separation strategy is implemented in the following example
of the function {\tt separate()}. Only if the pool separation fails, we
generate new cuts from scratch. The generated constraints are 
added with the function {\tt addCons()}
to the internal buffer, which has a limited size. The number
of constraints that can still be added to this buffer is returned
by the function {\tt addConBufferSpace()}\index{ABA\_SUB@{\tt ABA\_SUB}!addConBufferSpace@{\tt addConBufferSpace}}. 
The function {\tt mySeparate()} performs here the application specific 
separation. If more cuts are added with the function {\tt addCons()}
than there is space in the internal buffer for cutting planes, then
the redundant cuts are discarded. The function {\tt addCons()} returns
the number of actually added cuts.
\begin{verbatim}
  int MYSUB::separate()
  {
    int nCuts = constraintPoolSeparation();
    if (!nCuts) {
      ABA_BUFFER<ABA_CONSTRAINT*> newCuts(master_, addConBufferSpace());
      nCuts = mySeparate(newCuts);
      if (nCuts) nCuts = addCons(newCuts);
    }
    return nCuts;
  }
\end{verbatim}\index{ABA\_SUB@{\tt ABA\_SUB}!constraintPoolSeparation@{\tt constraintPoolSeparation}}
Note, \ABACUS\ does not automatically check if the added constraints are really
violated. Adding only non-violated constraints, can cause an infinite
loop in the cutting plane algorithm, which is only left if the tailing
off control is turned on (see Section~\ref{section:parameters}).

While constraints added with the function {\tt addCons()} are usually
allocated by the user, they are deleted by \ABACUS. They must {\bf
  not} be deleted by the user (see Section~\ref{section:memoryManagement}).

If not all constraints of the integer programming formulation are active,
and all discrete variables have integer values,
then the solution of a separation problem might be required to check
the feasibility of the LP-solution. In order to avoid a redundant call
of the same separation algorithm later in the function {\tt separate()},
constraints can be added already here by the function {\tt addCons()}.

In the following example of the function {\tt feasible()} the separation 
is even performed if there are discrete variables with fractional values
such that the separation routine does not have to be called a second
time in the function {\tt separate()}.
\begin{verbatim}
  bool MYSUB::feasible()
  {
    bool feasible;
 
    if (integerFeasible()) feasible = true;
    else                   feasible = false;
 
    ABA_BUFFER<ABA_CONSTRAINT*> newCuts(master_, addConBufferSpace());
      
    int nSep = mySeparate(newCuts);
 
    if (nSep) {
      feasible = false;
      addCons(newCuts);
    }
    return feasible;
  }
\end{verbatim}\index{ABA\_SUB@{\tt ABA\_SUB}!feasible@{\tt feasible}}

\subsubsection{Pricing out Inactive Variables}
\index{pricing}
\index{ABA\_SUB@{\tt ABA\_SUB}!pricing@{\tt pricing}}

The dynamic generation of variables is performed very similarly to the
separation of cutting planes. Here, the virtual function {\tt pricing()}
has to be redefined and the argument
{\tt pricing} in the constructor of the class {\tt ABA\_MASTER} should receive
the value {\tt true}, otherwise the pricing is skipped.

We illustrate the redefinition of the function {\tt pricing()} by an
example that is an analogon to the example given for the function 
{\tt separate()}.
\begin{verbatim}
  int MYSUB::pricing()
  {
    int nNewVars = variablePoolSeparation();
    if (!nNewVars) {
      ABA_BUFFER<ABA_VARIABLE*> newVariables(master_, addVarBufferSpace());
      nNewVars = myPricing(newVariables);
      if (nNewVars) nNewVars = addVars(newVariables);
    }
    return nNewVars;
  }
\end{verbatim}\index{ABA\_SUB@{\tt ABA\_SUB}!variablePoolSeparation@{\tt variablePoolSeparation}}

While variables added with the function {\tt addVars()} are usually
allocated by the user, they are deleted by \ABACUS. They must {\bf
  not} be deleted by the user (see Section~\ref{section:memoryManagement}).

\subsubsection{Primal Heuristics}
\index{primal heuristics}
\index{ABA\_SUB@{\tt ABA\_SUB}!improve@{\tt improve}}

\noindent
After the LP-relaxation has been solved in the subproblem optimization
the virtual function {\tt improve()} is called. Again, the default implementation
does nothing but in a redefinition in the derived class {\tt MYSUB} application
specific primal heuristics can be inserted:
\begin{verbatim}
  int MYSUB::improve(double &primalValue)
  { }
\end{verbatim}
If a better feasible solution is found its value has to be stored in {\tt primalValue}
and the function should return 1, otherwise it should return 0.
In this case, the value of the primal bound is updated by \ABACUS,
whereas the solution itself has to be updated within the function
{\tt improve()} as already explained for the function {\tt feasible()}.

It is also possible to update the primal bound already within the
function {\tt improve()} if this is more convenient to reduce internal
bookkeeping. In the following example we apply the two problem specific
heuristics {\tt myFirstHeuristic()} and {\tt mySecondHeuristic()}.
After each heuristic we check if the {\tt value} of the solution
is better than the best known one with the function call
{\tt master\_->betterPrimal(value)}\index{ABA\_MASTER@{\tt ABA\_MASTER}!betterPrimal@{\tt betterPrimal}}. 
If this function
returns {\tt true} we update the value of the best known feasible solution by
calling the function 
{\tt master\_->primalBound()}\index{ABA\_MASTER@{\tt ABA\_MASTER}!primalBound@{\tt primalBound}}.
\begin{verbatim}
  int MYSUB::improve(double &primalValue) 
  {
    int status = 0;
    double value;
  
    myFirstHeuristic(value);
    if (master_->betterPrimal(value)) {
      master_->primalBound(value);
      primalValue = value;
      status = 1;
    }
  
    mySecondHeuristic(value);
    if (master_->betterPrimal(value)) {
      master_->primalBound(value);
      primalValue = value;
      status = 1;
    }
  
    return status;
  }
\end{verbatim}

\subsubsection{Accessing Important Data}

\noindent
For a complete description of all members of the class {\tt ABA\_SUB} we refer
to the documentation in the reference manual.
However, in most applications only a limited number
of data is required for the implementation of problem specific functions,
like separation or pricing functions.
For simplification we want to state some of these members here:

\noindent
\begin{tabular}{lp{0.5\textwidth}}
{\tt int nCon() const;}   &returns the number of active constraints.\\
{\tt int nVar() const;}   &returns the number of active variables.\\
{\tt ABA\_VARIABLE *variable(int i);} &returns a pointer to the {\tt i}-th active
                                 variable.\\
{\tt ABA\_CONSTRAINT *constraint(int i);} &returns a pointer to the {\tt i}-th active
                                     constraint.\\
{\tt double *xVal\_;}                &an array storing the values
                                     of the variables after the
                                     linear program is solved.\\
{\tt double *yVal\_}                 &an array storing the values
                                     of the dual variables after the
                                     linear program is solved.\\
\end{tabular}


\subsection{Starting the Optimization}
\index{optimization}

After the problem specific classes are defined as discussed in the previous
sections, the optimization can be performed with the following main
program. We suppose that the master of our new application has as
only parameter the name of the input file.
\begin{verbatim}
  #include "mymaster.h"
  
  int main(int argc, char **argv)
  {
    MYMASTER master(argv[1]);
  
    master.optimize();
    return master.status();
  }
\end{verbatim}\index{ABA\_MASTER@{\tt ABA\_MASTER}!optimize@{\tt optimize}}
