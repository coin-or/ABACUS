% $Id: introduction.tex,v 2.9 2007-08-14 15:13:25 baumann Exp $

\chapter{Preface}

\section*{Preface to Release 3.0}

Major enhancements in \ABACUS~3.0 include the new solver interface to Osi, the ability to solve LPs with the Volume Algorithm and support for state-of-the-art GNU compilers. The documentation system has been changed from cweb to doxygen and the build process has been simplified.
ABACUS~3.0 is released under the Gnu Lesser General Public License (LGPL). 
See section~\ref{section:new30} for details. \newline
We thank the members of Michael J\"unger's group for many stimulating
discussions and valuable insights that helped improve the current
release. Special thanks go to Christoph Buchheim, Frauke Liers,
Thomas Lange (all University of Cologne) and Markus Chimani 
(University of Dortmund).
    

\bigskip
\bigskip\noindent
{K\"oln, August 2007 \hfill {\it Frank Baumann}, {\it Mark Sprenger} and {\it Andrea Wagner}}\newline

\section*{Preface to Release 2.3}

ABACUS~2.3 is the first commercial release of
\ABACUS\ distributed by the newly founded company Oreas GmbH. The
changes made involve mainly some bug fixes and a new
licensing mechanism, which allows to distribute also 30-days
evaluation licenses.


\bigskip
\bigskip\noindent
{K\"oln, December 1999 \hfill {\it Matthias Elf} and {\it Carsten
Gutwenger}}\newline
\smallskip
{\hfill Oreas GmbH}



\section*{Preface to Release 2.2}

\ABACUS\ is a software system for the implementation of \lpbab\
algorithms, i.e., \bac\ algorithms, \bap\ algorithms, and their
combination. It applies the concepts of object oriented programming
(programming language \CPLUSPLUS). An implementation of a problem
specific algorithm is obtained by deriving some classes from abstract
base classes of \ABACUS\ in order to embed problem specific functions.

Based on our earlier work on non-object oriented branch-and-cut
frameworks, Stefan Thienel developed \ABACUS~1.0 in his PhD thesis that
was defended in December~1995. Since January~1996 he developed the
public releases \ABACUS~1.2 to~2.1 with the partial support of 
ESPRIT LTR Project no.\ 20244 (ALCOM-IT) and
H.C.M.\ Institutional Grant no.\ ERBCHBGCT940710 (DONET).
Stefan Thienel laid the foundations of \ABACUS\ with great dedication
and enthusiasm. We regret that he decided to leave the Universit\"at zu
K\"oln in spring~1998. Very much to our satisfaction, Max B\"ohm and
Thomas Christof immediately took over the responsibility for
\ABACUS. We are very glad that \ABACUS\ is again in competent hands and future
development and maintenance is guaranteed.

\bigskip
\bigskip\noindent
{K\"oln, August 1998 \hfill {\it Michael J\"unger}}\newline
\smallskip
{Heidelberg, August 1998 \hfill {\it Gerhard Reinelt}}


\medskip

ABACUS~2.1 was left ready for release in February 1998 by Stefan Thienel.
After Stefan Thienel left university and we
took over the responsibility for \ABACUS, we decided
not to release ABACUS~2.1, but to add some new features to the software.
The major enhancements of the resulting version 2.2 are
the interface to the LP solver Xpress
and the compilation of \ABACUS\ with different native compilers. 
In addition, we introduced some new functions for easier parameter
handling and improved the HTML version of the Reference Manual.
A complete presentation of all modifications can be found in
Section~\ref{section:new22}.

We are very grateful to Stefan Thienel for his efforts involved in
the development, documentation and support of \ABACUS, and wish him
the very best for his future. For the users of \ABACUS, we hope that
this transition in responsibility will be almost invisible to them.


\bigskip
\bigskip\noindent
{K\"oln, August 1998 \hfill {\it Max B\"ohm}}\newline
\smallskip
{Athens, GA, August 1998 \hfill {\it Thomas Christof}}



\section*{Preface to Release 2.1}


The main purpose of version 2.1 of \ABACUS\ is the provision of some
bug fixes. However, there are also a few new features that are
explained in Section~\ref{section:new21}.


\bigskip
\bigskip\noindent
{K\"oln, February 1998 \hfill {\it Stefan Thienel}}


\section*{Preface to Release 2.0}

During its first year of public availability \ABACUS\ reached a rather
active community of users, which is growing slowly but
constantly. Many of them contributed to making \ABACUS\ more
reliable. I want to thank all of them for their helpful feedback.
In particular, I want to mention Max B\"ohm, who pointed me to several
improvement possibilities.

But not only the users worked with \ABACUS, also its development
continued such that it is now ready for a second release. \ABACUS~2.0
offers besides many minor extensions four major new features: 
\begin{itemize}
\item the interface to the new LP-solver SoPlex
\item the support of the Visual \CPLUSPLUS\ compiler
\item a generalized strong branching method
\item increased safety against name collisions
\end{itemize}
In particular, I am very happy that the abstract LP-interface proved
its usefulness during the integration of the LP-solver SoPlex. 
Since the adaption of the framework to the Visual \CPLUSPLUS\ compiler
could be performed, I am optimistic that also other compilers can be
supported in the future.

Users who want to upgrade from version~1.2.x find the new features and
the differences to previous versions in Section~\ref{section:NewFeatures}.

\bigskip
\bigskip\noindent
{K\"oln, August 1997 \hfill {\it Stefan Thienel}}

\section*{Preface to Release 1.2}

While the Chapters 1 to 4 of this manual are a user's guide describing
the installation, design, and application of \ABACUS\, the last 
chapter contains the reference manual.
Chapter~\ref{chapter:installation} explains how \ABACUS\ is installed
on your computer system and what hardware and software environment is
required. In order to simplify the user understanding \ABACUS\
I describe in Chapter~\ref{chapter:design} the design of the software
framework. While I recommend to study in any case the basic concepts outlined
in Section~\ref{section:DesignBasics} before beginning with the 
implementation of an application, it should be sufficient to return
to Section~\ref{section:DesignDetails} only for rather advanced
usage. Also Chapter~\ref{chapter:UsingAbacus} is split into two
sections. The first one, Section~\ref{section:UsingAbacusBasics},
explains the first steps that have to be performed to implement an
application. This section should be studied together with the example
included in the \ABACUS\ distribution. The second one, 
Chapter~\ref{section:UsingAbacusAdvanced}, shows how default strategies
of \ABACUS\ can be modified and outlines some additional features
of the system. The reference manual of 
Chapter~\ref{chapter:ReferenceManual} is complemented by the index that
simplifies finding a certain class or one of its members.

This manual is both available in Postscript and HTML format.
The HTML form turns out to be quite useful for finding members
of the reference manual.

This user's guide is not intended to teach the concepts of 
\lpbab, but I assume that the reader of this manual and the user
of \ABACUS\ is familiar with these algorithms. For an introduction
to \bac\ I refer to \cite{JRT95}, for an introduction to \bap\
algorithms I recommend to \cite{BJN97}. Both approaches are described
in \cite{Thi95}.

Moreover, I also assume that the user of \ABACUS\ is familiar with
the concepts of object oriented programming.
For the reader who is unexperienced in object oriented programming I refer to \cite{KM90} for
a good brief introduction and to \cite{Boo94} for a detailed
description. There are many books about the programming language
\CPLUSPLUS. The classical introduction is \cite{Str93a}. Very
useful reference manuals are \cite{ES92} and the current working
paper of the \CPLUSPLUS\ standardization committee \cite{Iso95}.

\ABACUS\ originates from the dissertation of its author \cite{Thi95}
and has since then been tested, slightly modified and
improved. Here, I would like to thank all initial testers, in
particular Thomas Christof, Meinrad Funke, and Fran\c{c}ois Margot
for their bug reports and helpful comments. I am very grateful to
Joachim Kupke for carefully proofreading an earlier version. I also
want to thank Denis Naddef, LMC-IMAG, Grenoble, France, for his
hospitality while writing the major part of this manual.

Despite these successful tests I consider \ABACUS\ still as an experimental
system. 
Therefore, feedback of the users is appreciated.
Some parts of the user's guide were adapted from \cite{Thi95}, while
the reference manual has been compiled for the first time. Therefore,
I also encourage the reader to send me error reports and improvement
suggestions for the user's guide and the reference manual.

I am aware that neither the software nor its documentation is perfect,
but I think it is time to dare a first public release.

\bigskip
\bigskip\noindent
{Grenoble, August 1996 \hfill {\it Stefan Thienel}}


