% $Id: referencemanual.tex,v 2.5 2007/08/08 17:04:39 baumann Exp $

\chapter{Reference Manual}
\label{chapter:ReferenceManual}

The reference manual covers only those classes and class members which
are relevant for the user. Therefore, the declarations of the classes
in this chapter contain only a subset of the actual members, e.g., 
private members are usually not documented here. For some classes
the copy constructor and/or assignment operator have not been defined,
but the default copy constructor and/or assignment operator are not
correct. In this case we declare this function and/or this operator
as a private member of its class such that its invalid usage is
detected already at compile time. In this reference manual this
is documented by including the copy constructor and/or assignment
operator in the private part of a function. Even if there are other
private members of the class they are not documented here.

This reference manual is automatically compiled from the source files
of \ABACUS. The advantage of this method is that we can always provide
an up to date version of the reference manual in future releases of
the software. The major drawback of this procedure is that the lack
of order of the functions in the current source files is reflected in
the reference manual. In particular, there is a often a difference of the
order of the member functions in the header of a class and in the 
documentation. For this reason we added HTML  links in
the declaration part of the class which point to other classes
and to the descriptions of the class members.
For a listing of all functions in lexicographical
order we refer to  the index.


At the end of the reference manual a list of all preprocessor flags is
given.

\section{Application Base Classes}

In order to implement an \ABACUS\ application problem specific classes
have to be derived from the classes {\tt ABA\_MASTER} and {\tt ABA\_SUB}. 
\ABACUS\ provides already some non-abstract classes derived from the
classes {\tt ABA\_CONSTRAINT} and {\tt ABA\_VARIABLE}, but if there is 
application specific structure to be exploited, classes also have 
to be derived from {\tt ABA\_VARIABLE} and {\tt ABA\_CONSTRAINT}.

Some other classes are included in this section because they are base
classes of the application base classes {\tt ABA\_MASTER}, {\tt ABA\_SUB},
{\tt ABA\_CONSTRAINT} and {\tt ABA\_VARIABLE}. The class {\tt ABA\_ABACUSROOT} is a
base class of every class of the system. The class {\tt ABA\_GLOBAL} is a 
base class of the class {\tt ABA\_MASTER}. Common features of constraints
and variables are embedded in the class {\tt ABA\_CONVAR}, from which the
classes {\tt ABA\_CONSTRAINT} and {\tt ABA\_VARIABLE} are derived.

\input class_a_b_a___a_b_a_c_u_s_r_o_o_t.tex

\input class_a_b_a___g_l_o_b_a_l.tex

\input class_a_b_a___m_a_s_t_e_r.tex

\input class_a_b_a___s_u_b.tex

\input class_a_b_a___c_o_n_v_a_r.tex

\input class_a_b_a___c_o_n_s_t_r_a_i_n_t.tex

\input class_a_b_a___v_a_r_i_a_b_l_e.tex

\input class_a_b_a___l_p_s_o_l_u_t_i_o_n.tex

\input class_a_b_a___s_e_p_a_r_a_t_o_r.tex

\%%%%%%%%%%%%%%%%%%%%%%%%%%%%%%%%%%%%%%%%%%%%%%%%%%%%%%%%%%%%%%%%%%%%%%%%%%%%%

\section{System Classes}

This section documents (almost) all internal system classes of
\ABACUS. This classes are usually not involved in the derivation
process for the implementation. However for retrieving special
information (e.g., about the linear program) or for advanced usage
we provide here a detailed documentation.

\input class_a_b_a___o_p_t_s_e_n_s_e.tex

\input class_a_b_a___c_s_e_n_s_e.tex

\input class_a_b_a___v_a_r_t_y_p_e.tex

\input class_a_b_a___f_s_v_a_r_s_t_a_t.tex

\input class_a_b_a___l_p_v_a_r_s_t_a_t.tex

\input class_a_b_a___s_l_a_c_k_s_t_a_t.tex

\input class_a_b_a___l_p.tex

\input class_a_b_a___o_s_i_i_f.tex

\input class_a_b_a___l_p_s_u_b.tex

\input class_a_b_a___l_p_s_u_b_o_s_i.tex

\input class_a_b_a___l_p_m_a_s_t_e_r.tex

\input class_a_b_a___l_p_m_a_s_t_e_r_o_s_i.tex

\input class_a_b_a___b_r_a_n_c_h_r_u_l_e.tex

\input class_a_b_a___s_e_t_b_r_a_n_c_h_r_u_l_e.tex

\input class_a_b_a___b_o_u_n_d_b_r_a_n_c_h_r_u_l_e.tex

\input class_a_b_a___v_a_l_b_r_a_n_c_h_r_u_l_e.tex

\input class_a_b_a___c_o_n_b_r_a_n_c_h_r_u_l_e.tex

\input class_a_b_a___p_o_o_l.tex

\input class_a_b_a___s_t_a_n_d_a_r_d_p_o_o_l.tex

\input class_a_b_a___n_o_n_d_u_p_l_p_o_o_l.tex

\input class_a_b_a___p_o_o_l_s_l_o_t.tex

\input class_a_b_a___p_o_o_l_s_l_o_t_r_e_f.tex

\input class_a_b_a___r_o_w.tex

\input class_a_b_a___c_o_l_u_m_n.tex

\input class_a_b_a___n_u_m_c_o_n.tex

\input class_a_b_a___r_o_w_c_o_n.tex

\input class_a_b_a___n_u_m_v_a_r.tex

\input class_a_b_a___s_r_o_w_c_o_n.tex

\input class_a_b_a___c_o_l_v_a_r.tex

\input class_a_b_a___a_c_t_i_v_e.tex

\input class_a_b_a___c_u_t_b_u_f_f_e_r.tex

\input class_a_b_a___i_n_f_e_a_s_c_o_n.tex

\input class_a_b_a___o_p_e_n_s_u_b.tex

\input class_a_b_a___f_i_x_c_a_n_d.tex

\input class_a_b_a___t_a_i_l_o_f_f.tex

\input class_a_b_a___h_i_s_t_o_r_y.tex

%%%%%%%%%%%%%%%%%%%%%%%%%%%%%%%%%%%%%%%%%%%%%%%%%%%%%%%%%%%%%%%%%%%%%%%%%%%%%%
\section{Basic Data Structures}

This subsection documents various basic data structures which we have used
within \ABACUS. They can also be used within an application. The
templated basic data structures are documented in 
Section~\ref{section:ReferenceTemplates}.

\input class_a_b_a___s_p_a_r_v_e_c.tex

\input class_a_b_a___s_e_t.tex

\input class_a_b_a___f_a_s_t_s_e_t.tex

\input class_a_b_a___s_t_r_i_n_g.tex

\section{Templates}
\label{section:ReferenceTemplates}

Various basic data structures are available as templates within
\ABACUS. For the instantiation of templates we refer to 
Section~\ref{section:UsingTemplates}.

\input class_a_b_a___a_r_r_a_y.tex

\input class_a_b_a___b_u_f_f_e_r.tex

\input class_a_b_a___l_i_s_t_i_t_e_m.tex

\input class_a_b_a___l_i_s_t.tex

\input class_a_b_a___d_l_i_s_t_i_t_e_m.tex

\input class_a_b_a___d_l_i_s_t.tex

\input class_a_b_a___r_i_n_g.tex

\input class_a_b_a___b_s_t_a_c_k.tex

\input class_a_b_a___b_h_e_a_p.tex

\input class_a_b_a___b_p_r_i_o_q_u_e_u_e.tex

\input class_a_b_a___h_a_s_h.tex

\input class_a_b_a___d_i_c_t_i_o_n_a_r_y.tex

\section{Tools}

This section documents some tools for sorting objects, measuring time,
and generating output.

\input class_a_b_a___s_o_r_t_e_r.tex

\input class_a_b_a___t_i_m_e_r.tex

\input class_a_b_a___c_p_u_t_i_m_e_r.tex

\input class_a_b_a___c_o_w_t_i_m_e_r.tex

\input class_a_b_a___o_s_t_r_e_a_m.tex

\section{Preprocessor Flags}
\label{section:preprocessor}

Table~\ref{table:preprocessor} summarizes all preprocessors flags that
are relevant for \ABACUS-users.

\begin{table}[htp]
\begin{center}
\newcommand{\cppflag}[1]{{\tt #1}\index{#1@{\tt #1}}}
\begin{tabular}{|l|l|l|}
\hline
Flag & Description & See Section \\
\hline
\cppflag{ABACUS\_COMPILER\_GCC41} & GNU \CPLUSPLUS\ compiler 4.1.x &
                                               \ref{section:CompilingAndLinking}  \\
\cppflag{ABACUS\_COMPILER\_GCC34} & GNU \CPLUSPLUS\ compiler 3.4.x &
                                               \ref{section:CompilingAndLinking}  \\
\cppflag{ABACUS\_COMPILER\_GCC33} & GNU \CPLUSPLUS\ compiler 3.3.x &
                                               \ref{section:CompilingAndLinking}  \\
\cppflag{ABACUS\_COMPILER\_SUN} & SUN C++ compiler & 
                                                \ref{section:CompilingAndLinking} \\
\hline
\end{tabular}
\caption{Preprocessor Flags.}
\label{table:preprocessor}
\end{center}
\end{table}
